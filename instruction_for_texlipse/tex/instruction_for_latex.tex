\documentclass{memoir}
\usepackage{fullpage}
\usepackage{enumitem}

% this file is presented for assignment in scientific writing class 
% MEEN 653-600 Scientific Writing fall2014
% 
% by:
% amir sohrabi


\begin{document}

 
\section*{Installing \LaTeX with TeXlipse on Windows 7 x64}
This instruction is prepared for someone who needs to produce high quality documents using \LaTeX.
He knows how to install Microsoft word on windows. 
He has access to internet and knows how to use browser. 
He knows how to download files from internet and how to run them.
He must have an administrative access to the computer he is installing TeXlipse onto.
Computer literacy in general is needed. 

The software that you will need: Tex Live, Eclipse. \\
Note: \LaTeX has two main distribution Tex Live and MiKTeX, we use Tex Live.  \\ 
Note: There are many editors for \LaTeX we use TeXlipse.  
\begin{enumerate}
  \item Download and install \LaTeX  distribution
\begin{enumerate}
  \item Go to website https://www.tug.org/TexLive/acquire-netinstall.html 
  \item Download and run install-tl-windows.exe
  \item Choose simple install (big) and click next
  \item Click install. This will unpack the installer into a temporary location. The installer will show up afterwards. Then click next
  \item Choose the mirror of your convenience from drop-down menu click next
  \item Choose the destination folder, remember this folder or write it down for future reference, then click next
  \item Choose the paper type that you will use for your documents then click next
  \item Check the installation setting or change them to your convenience and then click install
  \item Wait until installation completes this step will take some time depending on your system and internet connection
\end{enumerate}
\item Download and install Eclipse
\begin{enumerate}
  \item Go to https://www.Eclipse.org/downloads/ and choose your favorite Eclipse
  \item If in doubt which Eclipse you will need, click on windows 32Bit and download it
  \item Extract the downloaded files into c:\textbackslash  eclipse (this path step is necessaey since the path for some of the files in so long)
  \item Run Eclipse by double clicking on  c:\textbackslash  eclipse\textbackslash  eclipse.exe
\end{enumerate}
\item If faced a error asking for Java Runtime Enviroment follow next step if not disregard next step
\item Configure Java Runtime Enviroment for Eclipse
\begin{enumerate}
  \item Go to from http://www.oracle.com/technetwork/java/javase/downloads/java-se-jre-7-download-432155.html
  \item Accept the terms and down load JRE from http://download.oracle.com/otn-pub/java/jdk/7/jre-7-windows-x64.exe
  \item Install JRE after the download was complete
  \item Find JRE7 in c:\textbackslash  Program Files \textbackslash  Java
  \item Copy the JRE7 folder to c:\textbackslash  eclipse
  \item Rename the folder JRE7 to JRE (this will fix the problem for JRE because Eclipse will look for binary files in c:\textbackslash  eclipse \textbackslash jre \textbackslash bin)
\end{enumerate}
  \item Install TeXlipse inside Eclipse
  \begin{enumerate}
    \item Run Eclipse program 
    \item Go to the help menu, click on install new software
    \item Click on add (a dialogue box will show up)
    \item Choose TeXlipse as name
    \item Choose http://TeXlipse.sourceforge.net/ as location
    \item Click next
  \end{enumerate}
  \item Make the first Document
  \begin{enumerate}
    \item Run Eclipse
    \item Choose file in the menu 
    \item Choose TeXlipse project
    \item Write your document name
    \item Choose type of your document
    \item Choose format (Remember the location where you installed Tex Live.) 
    \item Click on setup build tools make sure all the tools have the .exe (binary) file allocated to them.
    \item If the build tools are empty, give the location of your Tex Live installation that you have saved in step 1 and click next
    \item Choose the name for that paths that you are going to work for hen click next
	\item Look for the text editor. There will be a space for you the document
    \item Type $\sum F = ma$ after $begin\{document\}$  
    \item Find your project's name on the left side in project explorer
    \item Open the project folder
    \item Open the output folder
    \item Double click on the pdf file that you have produced and see it    
  \end{enumerate}
  
  \item Finally, read, print, and review the document you have made with \LaTeX
\end{enumerate}

\end{document}
